\section{rsyslog}
\subsection{Überblick}
\begin{frame}[fragile]
	\frametitle{rsyslog}
	\framesubtitle{"Uberblick}
	\begin{itemize}
		\item Alternativer syslog-Daemon
		\item Standard bei Debian-basierten Distributionen
	\end{itemize}
	\begin{center}
		\begin{tiny}
			\begin{verbatim}
	2011-06-13T10:14:50 bayamo rsyslogd: -- MARK --
	2011-06-13T10:14:55 bayamo sshd[11015]: Accepted publickey for tm from 172.22.175.51 port 49425 ssh2
	2011-06-13T10:14:55 bayamo sshd[11015]: pam_unix(sshd:session): session opened for user tm by (uid=0)
	2011-06-13T10:14:56 bayamo sudo:       tm : TTY=pts/0 ; PWD=/home/tm ; USER=root ; COMMAND=/bin/su
	2011-06-13T10:14:56 bayamo su[11083]: Successful su for root by root
	2011-06-13T10:14:56 bayamo su[11083]: + /dev/pts/0 root:root
	2011-06-13T10:14:56 bayamo su[11083]: pam_unix(su:session): session opened for user root by tm(uid=0)
			\end{verbatim}
		\end{tiny}
	\end{center}
\end{frame}

\subsection{Vorteile}
\begin{frame}
	\frametitle{rsyslog}
	\framesubtitle{Vorteile}
	\begin{itemize}
		\item Bessere Sicherheitskontrolle
		\item Mehr Möglichkeiten für die Filterung
		\item Einfaches remote logging
		\item Protokollieren in die Datenbank
	\end{itemize}
\end{frame}

\subsection{Installation}
\begin{frame}[fragile]
	\frametitle{rsyslog}
	\framesubtitle{Installation}
	Paketmanager des Systems ist zu empfehlen
	\bigskip
	\begin{block}{Debian, Ubuntu}
		\begin{verbatim}
			apt-get install rsyslog
		\end{verbatim}
	\end{block}
\end{frame}

\subsection{Konfiguration}
\subsubsection{Konfigurationsdateien}
\begin{frame}[fragile]
	\frametitle{rsyslog}
	\framesubtitle{Konfigurationsdateien}
	\begin{block}{Dateien und Verzeichnisse}
		\begin{verbatim}
			/etc/rsyslog.conf
			/etc/rsyslog.d/*.conf
		\end{verbatim}
	\end{block}
	Sortierung erfolgt durch Nummerierung:
	\begin{itemize}
		\item \verb|/etc/rsyslog.d/00-AllowedHosts.conf|
		\item \verb|/etc/rsyslog.d/10-RemoteLinuxServers.conf|
		\item \verb|/etc/rsyslog.d/99-Default.conf|
	\end{itemize}
\end{frame}

\subsubsection{Remote Logging aktivieren}
\begin{frame}[fragile]
	\frametitle{rsyslog (server)}
	\framesubtitle{Remote Logging aktivieren}
	In \verb|/etc/rsyslog.conf|, folgende Zeile einfügen:
	\begin{verbatim}
		$ModLoad imudp
		$UDPServerRun 514
	\end{verbatim}
\end{frame}
\begin{frame}[fragile]
	\frametitle{rsyslog (Server)}
	\framesubtitle{Remote Logging aktivieren}
	Zugriffssteuerung in \verb|/etc/rsyslog.d/00-AllowRemoteLogging.conf|
	\begin{verbatim}
		# Ein Host
		$AllowedSender UDP, 192.168.56.100
		# Alle Hosts aus einem Subnetz
		$AllowedSender UDP, 192.168.56.0/24
		# Jeder von kernel.org
		$AllowedSender UDP, *.kernel.org
	\end{verbatim}
\end{frame}

\begin{frame}[fragile]
	\frametitle{rsyslog (Linux Client)}
	\framesubtitle{Remote Logging aktivieren}
	Eintrag in \verb|/etc/rsyslog.d/00-RemoteLogging.conf|
	\begin{verbatim}
		*.*		@192.168.56.1
	\end{verbatim}
	Viele syslog-Daemons werden unterstützt
\end{frame}
