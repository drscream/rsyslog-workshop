\section{tenshi}
\subsection{Überblick}
\begin{frame}[fragile]
	\frametitle{tenshi}
	\framesubtitle{"Uberblick}
	Tenshi ...
	\begin{itemize}
		\item ist ein Log Analyse Tool
		\item informiert
		\item wird als Dienst ausgeführt
	\end{itemize}
\end{frame}

\subsection{Vorteile}
\begin{frame}[fragile]
	\frametitle{tenshi}
	\framesubtitle{Vorteile}
	\begin{itemize}
		\item Einfach
		\item Schnelle Installation und Konfiguration
		\item Benutzerdefinierte Suchausdrücke
		\item Ziel und Zeit der Ausgabe ist flexibel
		\item E-Mail Benachrichtigung möglich
	\end{itemize}
\end{frame}

\subsection{Installation}
\begin{frame}[fragile]
	\frametitle{tenshi}
	\framesubtitle{Installation}
	Eine Installation ist mit dem Paketmanager möglich.
	\begin{block}{Debian, Ubuntu}
		\begin{verbatim}
			apt-get install tenshi
		\end{verbatim}
	\end{block}
\end{frame}

\subsection{Konfiguration}
\begin{frame}[fragile]
	\frametitle{tenshi}
	\framesubtitle{Konfigurationsdateien}
	\begin{block}{Grundeinstellungen}
		\begin{verbatim}
			/etc/tenshi/tenshi.conf
		\end{verbatim}
	\end{block}
	\bigskip
	\begin{block}{Ordner für Regeln}
		\begin{verbatim}
			/etc/tenshi/includes-active
			/etc/tenshi/includes-available
		\end{verbatim}
	\end{block}	
\end{frame}

\begin{frame}[fragile]
	\frametitle{tenshi}
	\framesubtitle{Logs / Mailserver}
	\begin{block}{Welche Dateien sollen "uberwacht werden?}
		\begin{verbatim}
			set logfile /var/log/auth.log
			set logfile /var/log/remote-logs/webserver1.log
		\end{verbatim}	
	\end{block}
	\bigskip
	\begin{block}{"Uber welchen Mailserver werden E-Mails versendet?}
		\begin{verbatim}
			set mailserver localhost
		\end{verbatim}	
	\end{block}
\end{frame}

\begin{frame}[fragile]
	\frametitle{tenshi}
	\framesubtitle{Message - Queues}
	Message-Queues ... 
	\begin{itemize}
		\item sind Verteiler
		\item können verschiedene Empfänger haben
		\item haben einen flexiblen Intervall
	\end{itemize}
	\bigskip
	\begin{block}{Syntax}
		\begin{scriptsize}
			\begin{verbatim}
				set queue <queue_name> <mail_from> <mail_to><interval>

				set queue foo SrcMail DstMail [now] Server crushed !!!
				set queue bar SrcMail2 DstMail2 [0 9 -17/2 * * *] Important !
			\end{verbatim}
		\end{scriptsize}
	\end{block}
\end{frame}

\begin{frame}[fragile]
	\frametitle{tenshi}
	\framesubtitle{Gruppen}
		Gruppen ...
	\begin{itemize}
		\item beinhalten Regeln
		\item verkürzen die Suchdauer
		\item werden in den Ordnern abgelegt \\ 
			\verb|/etc/tenshi/includes-active|\\
			\verb|/etc/tenshi/includes-available|
		\item Beispielgruppen sind vorhanden (loginsusudo, ssh, ...)
	\end{itemize}
\end{frame}

\begin{frame}[fragile]
	\frametitle{tenshi}
	\framesubtitle{Ein Beispiel}
	\begin{itemize}
		\item Auszug aus der Datei \verb|auth.log|
		\begin{block}{}
		\begin{tiny}
			\begin{verbatim}
Jun 1 17:19 loki su[1768]: Successful su for root by jowhite
Jun 1 17:19 loki su[1768]: + /dev/pts/1 jowhite:root
Jun 1 17:19 loki su[1768]: pam_unix(su:session): session opened for user root by jowhite(uid=1000)
Jun 1 17:19 loki su[1768]: pam_unix(su:session): session closed for user root
			\end{verbatim}
		\end{tiny}
		\end{block}
		\item Definition der Gruppe
		\begin{block}{}
		\begin{tiny}
			\begin{verbatim}			
group ^su\(pam_unix\):
su      ^su: pam_unix\(su:session\): session opened for user root(?: by \(uid=.+\))?
su      ^su: pam_unix\(su:session\): session closed for user root
su      ^su: Successful su for .+ by .+
group_end
			\end{verbatim}
		\end{tiny}
		\end{block}
	\end{itemize}
\end{frame}