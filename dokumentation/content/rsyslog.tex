\section{rsyslog}
	\subsection{Überblick}
		Rsyslog ist ein weiterer syslog-Daemon für Linux beziehungsweise Unix-Systeme. Er bietet, durch 
		die Möglichkeit der inhaltsbasierten Filterung, eine gute Grundlage für jedes System.
		In diesem Abschnitt wird die Installation und Konfiguration von rsyslog für den Remote Logging
		Einsatz vorgestellt.

		\Hinweisbox{
			Es wird rsyslog in der Version 4.6.4-2 unter Ubuntu Server 10.04 LTS eingesetzt. 
			Die Konfiguration kann daher unter anderen Versionen und Distributionen abweichen.
		}

	\subsection{Alternativen}
		Um einen besseren Überblick über die syslog-Landschaft zu erhalten werden einige Alternativen vorgestellt.
		\subsubsection*{syslogd}
		Bei syslogd handelt es sich um den ältesten und bekanntesten Protokoll-Daemon.\cite{GentooProtokollierung}
		\subsubsection*{syslog-ng}
	\subsection{Remote logging}
		\subsubsection*{Vorteile}
		\subsubsection*{Nachteile}
	\subsection{Installation}
		In einigen Linux Distributionen wie zum Beispiel Debian und Ubuntu ist rsyslog schon zum Standard 
		geworden. Für andere Distributionen empfiehlt es sich den Paketmanager zur Installation zu verwenden.

		\begin{tabular}{ll}
			\hline
			Linux Distribution & Installationsbefehl\\
			\hline\hline
			Debian, Ubuntu & apt-get install rsyslog\\
			SUSE, openSUSE & yast -i rsyslog\\
			Fedora         & yum install rsyslog\\
			Gentoo         & emerge rsyslog
		\end{tabular}

		\Hinweisbox{
			Gentoo Benutzer sollten die passenden Useflags für das rsyslog Paket beachten. Empfehlenswert sind
			\textit{gnutls}, \textit{logrotate} und \textit{zlib}. 
		}

	\subsection{Konfiguration}
		\subsubsection*{Server}
		\subsubsection*{Linux Client}
		\subsubsection*{Windows Client}
		\subsubsection*{Debugging}

