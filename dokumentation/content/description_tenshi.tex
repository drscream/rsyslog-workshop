\section{tenshi}
\subsection{Überblick}
Das Open-Source-Programm \textit{Tenshi} wurde vom Autor Andrea Barisani entwickelt und ist für eine Unix-Umgebung gedacht. Er lehnte sich dabei an das Programm \textit{Oak} an und schrieb es mit PERL (PEARL =\textbf{P}ractical \textbf{E}xtraction \textbf{a}nd \textbf{R}eport \textbf{L}anguage) um. Dieses Programm lässt sich mit RSysLog leicht kombinieren. Sinn dieser Software ist es, die ankommenden Systemlogdateien anderer Systeme zu Analysieren und den Systemadministrator möglichst gut und schnell zu Informieren wenn kritische Ereignisse auf einem System passiert sind. Dabei bleibt das Programm leicht zu konfigurieren und ist flexibel.

\subsection{Einführung}
Zur Überwachung von Log-Dateien ist Tenshi (früher Wasabi) gut geeignet. Es werden reguläre Ausdrücke vom Benutzer definiert und verschiedenen Message-Queues zugeordnet. Das Programm durchsucht die ausgewählten Log-Dateien Zeile für Zeile nach passenden Zeichenfolgen. Ist ein Treffer erzielt, so wird die gefundene Zeile in die dazu gehörige Message-Queue eingefügt. Je nach eingestellten Parametern kann die Message-Queue sofort eine E-Mail an den Empfänger schicken oder erst nach einem bestimmten Zeitintervall. 
Zur besseren Überschaubarkeit der gefundenen Zeilen und Reporte können die Standard-Gruppierungsoperatoren der regulären Ausdrücke verwendet werden, um irrelevante Felder auszublenden. 
Die Reporte sind nach Hostnamen sortiert und werden nach Möglichkeit komprimiert. \cite{TenshiDescription}

\subsection{Alternativen}
Für eine Windows-Umgebung  eignet sich das Programm \textit{Event-Watch} von Jürgen Auer. Es basiert auf dem .NET Framework und kann kostenlos geladen werden \url{http://www.sql-und-xml.de/freeware-tools/evt-watch.2.0.zip}. Die Konfiguration findet  in der \textit{evt-Watch.exe.xml}-Datei statt. An dieserstelle sind Kenntnisse in der XML (\textbf{E}xtensible \textbf{M}arkup \textbf{L}anguage) – Sprache erforderlich. Die erforderliche Syntax wird in der Datei \textit{evt-Watch.exe.xml } mithilfe von Kommentaren erklärt. Dieses Programm kann wie Tenshi mit Regulären Ausdrücken arbeiten um Filter auf die unzähligen Zeilen der Log-Dateien anzuwenden. Die Möglichkeit Berichte per E-Mail zu versenden ist auch hier realisiert. Eine kleine Erweiterung ist die Verwendung einer SQL-Datenbank für die Sicherung von Berichtsdaten.

