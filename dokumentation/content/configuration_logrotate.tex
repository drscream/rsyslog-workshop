\section{logrotate}
\subsection{Installation}
Bei den meisten Linux Distribution und rsyslog-Paket wird logrotate schon als Abhängigkeit installiert. Sollte dies nicht der Fall sein kann das Paket per gewünschtem Paketmanager nachinstalliert werden.

\begin{tabular}{ll}
	\hline
	Linux Distribution & Installationsbefehl\\
	\hline\hline
	Debian, Ubuntu & apt-get install logrotate\\
	SUSE, openSUSE & yast -i logrotate\\
	Fedora         & yum install logrotate\\
	Gentoo         & emerge logrotate\\
\end{tabular}

\subsection{Konfiguration}
In \textit{/etc/logrotate.conf} befindet sich die Standardkonfiguration, diese importiert alle weiteren Konfigurationsdateien aus \textit{/etc/logrotate.d}. Der Aufbau dieser Dateien sieht folgendermaßen aus.

\begin{lstlisting}
/pfad/zu/einer/oder/mehrerer/log/dateien.log {
	#	
	# Konfigurationsoptionen
	#
	postrotate
		#		
		# Script das nach Beendigung ausgef"uhrt werden soll
		#
	endscript
}
\end{lstlisting}

Da im vorherigen Abschnitt eine Verzeichnis \textit{/var/log/rsyslog.remote.archive} angelegt wurde. Ist nun gewünscht jeden Tag die alten Remote Protokolldateien zu archivieren. Die jeweilige Beschreibung der Konfigurationsoptionen erfolgt in der Datei.

\begin{lstlisting}
/var/log/rsyslog.remote/*.log {
	## Speicherung der rotierenden Log-Dateien in einem separaten Ordner
	olddir /var/log/rsyslog.remote.archive
	## T"agliche Rotation
	daily
	## Behalte die Log-Dateien 30 Tage
	rotate 30
	## Benutze die Formatierung .YYYYMMDD anstatt .0,.1,.2 etc.
	dateext
	## Rotation auch von leeren Log-Dateien
	ifempty
	## Erstelle die Quell-Log-Dateien erneut, mit den alten Rechten
	create
	## Packe die Log-Dateien immer erst einen Tag nach dem aktuellen
	## Durchlauf
	delaycompress
	## Packe Log-Dateien
	compress
	compresscmd /bin/gzip
	compressoptions -9
	uncompresscmd /bin/gunzip
	## Keine Fehlermeldung falls eine Datei nicht existiert
	missingok
	## Lade rsyslog Konfigurationen nach dem beenden von logrotate neu
	sharedscripts
	postrotate
		/etc/init.d/rsyslog reload &>/dev/null || true
	endscript
}
\end{lstlisting}